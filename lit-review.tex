\chapter{In defense of detectability: Importance of detection probability in bird surveys and recent advances in estimation}

\section{Introduction}

\par Detectability can be defined as the probability of observing an individual at a specific study site during a particular survey, given the individual is present.
Here, I specifically condition that the species is present to disentangle those factors that affect occupancy.
That is to say, we are only concerned with the probability of observing a species given it is there, rather than the joint probability of observing the species and the probability of the species being present.

\par In general, we can think of detectability as a function $f(\cdot)$ that scales the \textit{true} abundance of a species at a site down to the \textit{observed} abundance of a species at a site (and vice versa). 
That is, $n = Nf(\cdot)$, where $n$ is the observed abundance of a species, $N$ is the true abundance of the species, and $f(\cdot)$ is the detectability function where the range of $f(\cdot) \in [0,1]$.
Assuming no false positives, $n$ is necessarily a subset of $N$.

\par The detectability function $f(\cdot)$ can be further broken down into two components: availability and perceptibility \citep{marsh_correcting_1989}.
Here, we define availability $p(\cdot)$ as a function that generates the probability that a bird occupying a site gives a cue (such as a song, chip, or visual).
Then, we can define perceptibility $q(\cdot)$ as a function that generates the conditional probability that a bird giving a cue is perceived and recorded by observer.
Hence, we have that detectability $f(\cdot) = p(\cdot) \times q(\cdot)$.
Often, the terms ``perceptibility" and ``detectability" are used interchangeably (see for example CITATION CITATION CITATION).
However, for the purposes of this review, I will specifically use perceptibility when referring to the conditional probability that a cueing bird is recorded, whereas I will use ``detectability" when referring to the product of availability and perceptibility.
In additional, for the purposes of the following arguments in this paper, I will consider the observer to be a human conducting a point count or transect. However, at the end of this review, I will dive into opportunities for passive acoustic monitoring to function as an observer.

\par Detectability in landbirds is generally non-constant. 
Factors such as time of day, time of year, habitat type, and presence of roads, among several other factors, have been shown to affect both the availability and the perceptibility of birds \citep{wilson_reliability_1985, solymos_calibrating_2013, johnston_species_2014, cooke_road_2020}. 
Additionally, the length of time for which an observer surveys for a bird, and the maximum survey distance for which the observer is surveying, can account for some variation in how many birds are detected and recorded for any given survey \citep{alldredge_factors_2007, solymos_calibrating_2013, buckland_distance_2015}.  
Our goal in detectability research is to appropriately model the components $p(\cdot)$ and $q(\cdot)$ so as to derive an accurate and precise estimate of detectability that captures the variation.
Table 1 of \citet{johnson_defense_2008} nicely captures the various factors that affect both availability and perceptibility in birds.
While all of these factors still hold true, we choose to expand on the table, both by elaborating on many of the factors, and by attempting to further separate the factors into ``survey-specific" and ``species-specific" factors.
We feel this distinction is important when considering modelling variation in detectability both within and among species, in a variety of survey contexts.

\section{What affects availability?}

\par Recall that availability is the probability that a bird that is occupying a survey site gives a cue, such as a song, chip, or a visual cue during a survey.
Factors that affect availability will be those factors that affect both a) the birds propensity to give a cue, and b) a survey's ability to pick up a cue in time.

\subsection{Species-specific Factors}
\subsubsection{Phenology}
\par One of the most well-known factors affecting availability is time of day.
In most landbirds species, availability will peak (i.e., probability of availability will be close to 1) at dawn, or shortly after sunrise, during the ``dawn chorus".
This is the period of the day where many songbirds are active and most vocal.
Many surveys of landbirds take advantage of this peak availability by conducting surveys during these peak hours, typically from 1 hour before sunrise until approximately 3 to 4 hours after sunrise.

\par In addition to the time of day effect, time of year will also have an obvious effect on probability of availability.
Songbirds will vocalize, call, and appear more during peak territory establishment season, and even during migration,  than later on in the breeding season when the birds become more inconspicuous.
Similar to time of day, bird surveys are often conducted to try to hit this maximal availability in time of year.

\par These temporal effects may be confounded spatially, because (at least in North America) certain parts of the continent have an earlier spring than other parts of the continent.
In western North America, the ``green-up" time generally happens earlier, and so wide-spread species in North America may begin to sing earlier in the western part of the continent than the eastern part of the continent.
However, there is yet to be a study that clearly shows this, so this may be an avenue for future research.

\par It should be noted that while many songbirds fit into a classical ``peak availability in the morning and during early breeding season", it is not necessarily the case that all \textit{landbirds} fit into this description.
One clear example would be nightjars, where their typical displays (and therefore peak availability) are in the crepuscular hours.
This is also true of some shorebirds such as American Woodcock, where they are much less available during the day than in the evening hours.
As such, when modelling availability in landbirds in general (see discussion later), it may not necessarily be appropriate to throw a ``one size fits all" when considering temporal factors.

\subsection{Evolutionary Processes}
TO DO: This section is hot garbage
\par There is evidence that evolutionary processes may play a part as to when a bird starts singing in a year, and how often it sings. 
\citet{solymos_phylogeny_2018} showed that phylogeny can explain some of the variation in cue rates (i.e., the rate at which a bird calls per unit time, which contributes to its measure of availability in certain modelling approaches) among species. 
This may partly be because of how evolutionary processes contributed to how song works in birds (GARBAGE).
Additionally, traits such as whether or not a bird is migratory may also contribute to availability during a survey.
Birds that are migratory tend to sing at higher rates than birds that are resident (CITATION), because there is ``more to lose" for birds that migrate.

\par One newer factor that has occured since 2020 is the brief but signficant period of the "anthropause".
Some studies showed that birds may have actually been affected by this anthropause when less humans were out and about, in that birds tended to sing (CHECK WHAT HAPPENED HERE WITH WTSP IN CALIFORNIA).
This may suggest ongoing research to determine effects of anthropogenic noise on bird singing rate, and bird singing times.

\section{What affects perceptibility?}

\par Recall that perceptibility is the probability that a bird is recorded by an observer, given the bird has given a cue.
Unlike availability, which is generally driven by biological and temporal processes (as well as interactions between those processes), perceptibility is driven significantly by environmental factors, observer factors, biological factors, and to a much lesser extent, temporal factors.
This is because perceptibility is primarily concerned about all those factors that determine whether or not a cueing bird will accurately be recorded.
Note also that perceptibility contributes heavily to issues surrounding survey sensitivity, where we are concerned with lowering the false negative (i.e., missed individuals) rate.

\subsection{Landscape and Landcover Types}

\par In perceiving cueing birds, landcover type contributes the most to overall perceptibility.
At the very basic level, an environment with many trees (i.e., a more ``closed" environment) will contribute to a lower perceptibility for birds than an environment with fewer trees (i.e. a more ``open" environment) (CITATION).
This is because sound is absorbed (or ``attenuates") more in a closed environment, and therefore the sound level is diminished more by the time it reaches the observer.
Of course, visual identifications of birds can be much easier in an open environment compared to a closed environment as well.
In particular, grassland environments can allow for a much greater detection distance of soaring birds and displaying birds, while forested environments often prevent many visual sightings from further than one hundred or so metres.
Both of these basic factors result in detection distances that are smaller in closed environments than in open environments \citep{edwards_point_2023, solymos_calibrating_2013, yip_sound_2017}.

\par Within an ``open" or ``closed" environment, numerous factors about the environment itself can contribute to changes in detection distance and therefore perceptibility.
There is some evidence to suggest that different types of forest (e.g., broadleaf, needleleaf, mixed forest, etc.) can affect the attenuation of a bird's song, in that broadleaf forests tend to diminish sound quicker than needleleaf forests \citep{schieck_biased_1997}.
In both open and closed environments, wind can have a significant interacting effects on perceptibility.
In an open environment, wind can ``carry" sound from one direction while diminishing sound from the opposite direction, creating a deformed circle of detection \citep{rigby_factors_2019}, whereas in a closed environment, wind can have a masking effect due to added tree noise (which is also an effect of the type of forest) (CITATION).
Finally, within closed environments, maturity of the forest can have a significant effect on sound attenuation, in that forests with larger trees (measured by metrics such as diameter at breast height or DBH) generally will have greater sound attenuation than less mature forests containing trees with smaller DBH.
This has implications for setting up ARUs in forested areas, particularly for accounting for attenuation in localization grids (CITATION).

\subsection{Roads}

\par Roads provide a unique challenge in estimating detection distance and hence perceptibility for birds, in that there are often opposing effects that contribute to an overall perceptibility estimate.

\par In general, we expect to have an increased perceptibility of birds during a roadside survey compared to an offroad survey, because the hard surface of the road allows for sound to ``bounce" and travel further.
This is, of course, in addition to the fact that a road is effectively an open habitat, which already would tend to increase perceptibility of a bird singing near a road.
However, given the linearity of roads, this effect does not occur uniformly across the survey radius.
That is, rather than a circular effective detection radius for a singing bird, we may expect a elliptical detection shape, whereby the major axis of the ellipse follows the road and the minor axis is perpendicular to the road.
The shape of this ellipse would then also be affected by the habitat perpendicular to the road, in that a forested habitat would generate a detection shape with a much smaller minor axis than an open habitat would (FIGURE WOULD BE USEFUL).
Regardless, the hard surface of the road and the open habitat that a road creates will generally increase perceptibility compared to an offroad survey.

\par On the other hand, road noise may have opposite effects on perceptibility.
\citet{cooke_road_2020} found, in a study in the UK, that road noise tended to decrease the detectability of birds, particularly for birds that are smaller in body size and therefore rely heavier on aural cues for detection.
However, some species showed an improved detectability, further contributing to the nuances of roads.
In North America, there is some evidence that higher traffic rates contributes to decreases in bird abundance at different BBS sites \citep{griffith_traffic_2010}.
These effects were observed across a suite of bird species that ranged in both abundance and song frequency.

\par More work is needed to assess the effects of traffic on detectability \textit{per se} versus detectability as a function of occupancy and/or abundance.
Obviously, roads represent a loss of habitat via a linear feature in the environment.
However, different species respond in different ways to these linear features.
Some species such as TO DO and TO DO are attracted to linear features, particularly roads, whereas others may be driven away from these features.
Species that are attracted to linear features may therefore have a lower estimated detection distance in a roadside environment because the observer records them at the road closer to the survey centroid.
More research should be done to truly estimate how these linear features affect detectability.


\subsection{Traits}

\par Traits contribute a significant proportion of what drives perceptibility in landbirds.
In particular, certain traits such as body size and song pitch contribute greatly to how easily a bird can be seen or heard during a survey.
In most cases, a bird with a larger body size will be easier to detect than a bird with a smaller body size, and thus will have a higher perceptibility.
Similarly, a bird with a higher-pitched song will be more difficult to hear than a bird with a lower-pitched song (to an extent), and will therefore have a lower perceptibility.
This also interacts with observer effects (see subsection Observer Effects) when considering the potential effects of hearing loss among observers as they age.

\par In addition to song pitch, song type can play a major role in the perceptibility of a bird.
This is particularly true when considering interactions with the environment.
Indeed, there is evidence to suggest that bird song has evolved partially depending on their preferred habitat.
In forested environments, pure-tone-based songs, such as those from White-throated Sparrow (\textit{Zonotrichia albicollis}), can carry better through the excess attenuation of the forest (CITE Morton and Brumm citations), whereas in an open environment, birds are more likely to have "rattle" songs (CITE Wiley 1991).
As such, habitat preference can be a factor that affects perceptibility, in that it could be used as a proxy for these song types.

\section{Modelling of Detectability}

\par In general, at any given sampling event, for species $s$, there exists the true species abundance $Y_s$, which represents all the individuals of species $s$ at that site at the time of the sampling event. During any sampling event, the recorded count for species $s$ (i.e., $\hat{Y_s}$), is a subset of the true abundance $Y_s$, such that $\hat{Y_s} \leq Y_s$, assuming individuals are correctly identified \citep{bennett_how_2024, johnson_defense_2008}. The difference between the recorded count and the true count depends on the detectability of the birds. The factors that affect detectability of species $s$ can be accounted for in the latent detectability function $f(\cdot)$, such that
\begin{equation}\label{detect-function}
	\mathbb{E}\left[\hat{Y_s}\right] = Y_s \times f(\cdot).
\end{equation}

\par Let $p_s(t)$ be the availability of species $s$ for a given survey duration $t$, and let $q_s(r)$ be the perceptibility of species $s$ for a given survey radius $r$ from an observer \citep{solymos_calibrating_2013}. The detectability function can now be written as $f(\cdot) = p_s(t) \times q_s(r)$, and Equation \ref{detect-function} can now be written as:

\begin{equation*}
	\mathbb{E}\left[\hat{Y_s}\right] = Y_s \times p_s(t) \times q_s(r).
\end{equation*}

\par $p_s$ is calculated from surveys that employ a removal sampling approach \citep{alldredge_time--detection_2007, farnsworth_removal_2002, solymos_evaluating_2018}, in which total survey time is partitioned into “time bins” $t_J$, not necessarily of equal size. For example, a 6-minute survey may be partitioned into $J = 6$ 1-minute time bins. Or, it could be partitioned into a 1-minute time bin, followed by a 3-minute time bin, followed by a 2-minute time bin, for a total of $J = 3$ time bands. If a bird gives a cue and is recorded in time bin $j$, then it is “removed” from the available birds to be recorded in all subsequent time bin. Cue rates follow a Poisson process with Poisson parameter $\phi$, which is estimated via maximum likelihood estimation or Bayesian estimation given survey type and any number of covariates, and the time to first cue event (i.e., detection) follows an exponential distribution $f(t) = \phi e^{-t\phi}$. Thus, the cumulative distribution function for time to first detection on the interval $\left[0, t_J\right]$ is given by (Alldredge et al., 2007)
\begin{align*}
	p(t_J) &= \int_{0}^{t_J} \phi e^{-t\phi} dt \\
	&= 1 - e^{-t\phi}.
\end{align*}

\par $q_s$ is calculated from surveys employing a distance sampling protocol \citep{buckland_introduction_2001, buckland_distance_2015}. Distance sampling partitions the point count survey area into discrete distance bins $r_K$. For example, when considering a 200m point count radius, the area may be partitioned into $K = 2$ 100m bins, or $K = 4$ 50m bins. The outermost bin can have finite or infinite truncation distance. Birds are then recorded by the observer as to how far away from the observer the bird was perceived. \citet{solymos_calibrating_2013} assume a half-normal detection function with probability of detecting an available bird at a distance $r$ from the observer is $g(r) = e^{-\dfrac{r^2}{\tau^2}}$, where $\tau^2$ is the variance of the unfolded normal distribution that describes the rate of distance decay and is estimated via multinomial conditional maximum likelihood estimation given survey type and any number of covariates. Following \citet{buckland_introduction_2001, solymos_calibrating_2013} the probability $q_s$ that an individual bird located within radius $r_K$ gives a cue that is perceived by an observer is given by
\begin{equation*}
	q(r_K) = \dfrac{\pi \tau^2 \left(1 - e^{-\dfrac{r^2}{\tau^2}}\right)}{\pi r^2_K}
\end{equation*}

\subsection{The QPAD Approach to Detectability Estimation}

\par The QPAD method developed by the Boreal Avian Modelling project is a flexible approach to accounting for heterogeneity in survey conditions and survey methodology \citep{solymos_calibrating_2013}. It can calculate availability and perceptibility independently, while allowing for multiple surveying methods to be accounted for at once. In other words, any dataset that employs a removal sampling approach with two or more time bins can be jointly used to calculate availability, and any dataset that employs a distance sampling approach with two or more distance bins can be jointly used to calculate perceptibility (Sólymos et al. 2013). Additionally, by recognizing that availability is a function of cue rate, and that perceptibility is a function of EDR, the QPAD method allows for variation in cue rate and EDR as a function of covariates that affect detectability (such as time of day, time of year, habitat type, roadsides, etc.), and for estimates of perceptibility as a function of survey radius or other covariates (Sólymos et al. 2013). 

\par For any point count that employed removal sampling \citep{alldredge_time--detection_2007, farnsworth_removal_2002} and/or distance sampling \citep{buckland_introduction_2001, buckland_distance_2015}, let $\hat{Y}_{sijk}$ be the observed count of species $s$ during sampling event $i$, occurring in time band $j \in [1,J]$ and/or distance band $k \in [1,K]$.
That is, if we have a point count where removal sampling or distance sampling (but not both) was not used, then either $J = 1$ or $K = 1$, respectively; whereas, if we have a point count where both removal sampling and distance sampling was used, then both $J,K \geq 2$.
The case where $J = K = 1$ is where neither removal sampling nor distance sampling was used, and so these data points are not appropriate for the modelling techniques described here.

\par For QPAD removal modelling, we are only interested in the time bin in which the bird was recorded, and not the distance, and so we sum counts from each sampling event $i$ over all distance bands. 
Thus, we consider counts $\hat{Y}_{sij.} = \sum_{k=1}^{K} \hat{Y}_{sijk}$. 
From \citet{solymos_calibrating_2013}, we model $\hat{Y}_{sij.}$ as

$$\hat{Y}_{sij.} \sim multinomial\left(\hat{Y}_{si..}, \mathbf{\Pi}_{si.}\right).$$

\par For sample $i$, $\mathbf{\Pi}_{si.}$ is the corresponding probability vector that determines how the total number of species $s$ observed at sample $i$ (i.e., $\hat{Y}_{si..}$) are to be distributed across the $J$ time bands; in essence, it is the mixing parameter.
Let $t_{ij}$ be the maximum time for time band $j$ during sampling event $i$, and let $\phi_s$ be the unknown cue rate for species $s$.
We then have the following calculation of each component $\pi_{sij}$ of the mixing parameter $\mathbf{\Pi}_{si.}$:

\begin{equation*}
	\pi_{sij} = 
	\begin{cases}
		\dfrac{\exp\left\{ -t_{i,j-1}\phi_{s} \right\} - \exp\left\{ -t_{ij}\phi_{s} \right\}}{1 - \exp\left\{ -t_{iJ}\phi_{s} \right\}} & \text{for } j > 1 \\
		1 - \sum_{n = 2}^{J} \pi_{sin} & \text{for } j = 1
	\end{cases}
\end{equation*}

\par From here, we would then be interested in estimating $\phi_s$, the cue rate for species $s$.
This would be done in the form of a regression model, where we would have $\phi_s$ varying based on any combination of the factors that affect availability through cue rate, such as time of day, time of year, etc.
This can be done using the \texttt{detect} R package \citep{solymos_detect_2020}.

\par For distance modelling, we are only interested in the distance to the cueing bird, and not the time bin that the bird was recorded in, and so we sum counts from each sampling event $i$ over all time bands. 
Thus, we consider counts $\hat{Y}_{si.k} = \sum_{j=1}^{J}\hat{Y}_{sijk}$. 
From \citet{solymos_calibrating_2013}, we model $\hat{Y}_{si.k}$ as
$$\hat{Y}_{si.k} \sim multinomial\left(\hat{Y}_{si..}, \mathbf{\Pi}_{si.}\right).$$

\par For sample $i$, $\mathbf{\Pi}_{si.}$ is the corresponding probability vector that determines how the total number of species $s$ observed at sample $i$ (i.e., $\hat{Y}_{si..}$) are to be distributed across the $K$ distance bins; in essence, it is the mixing parameter. 
Let $r_{ik}$ be the maximum distance for distance bin $k$ during sampling event $i$, and let $\tau_s$ be the unknown effective detection radius for species $s$. 
We then have the following calculation of each component $\pi_{sik}$ of the mixing parameter $\mathbf{\Pi}_{si.}$:

\begin{equation*}
	\pi_{sik} = 
	\begin{cases}
		\dfrac{f(r_{i,k}, \tau_s) - f(r_{i,k-1}, \tau_s)}{f(r_{i,K}, \tau_s)} & \text{for } k > 1 \\
		1 - \sum_{n = 2}^{K} \pi_{sin} & \text{for } k = 1
	\end{cases}
\end{equation*}

where 
$$f(r,\tau) =  1 - \exp\left\{ -\dfrac{r^2}{\exp\left\{\tau^2\right\}} \right\}.$$

\par As with QPAD removal modelling, for QPAD distance modelling we are most interested in estimating $\tau_s$, the effective detection radius for species $s$. 
This would also be done in the form of a regression model, where we would have $\tau_s$ varying based on any combination of the factors that affect perceptibility through EDR, such as roads, landcover, etc.
This can also be done using the \texttt{detect} R package \citep{solymos_detect_2020}.

\par The QPAD method allows for estimates of true density to be derived from any survey, by allowing the detectability function to act as a statistical offset to account for differences among survey types. An offset term is used in linear models to adjust the expected value with a known quantity. In our case the detectability function quantity is not known but estimated through QPAD. But as a result, the offsets allow all survey-observed counts to be translated into an estimate of true density. That is, by recognizing that $Y_s = D_s \times A$, where $D_s$ is the density of species $s$ and $A$ is the total area sampled at a point count, we can get that $\mathbb{E}\left[\hat{Y_s}\right] = D_s \times A \times p_s(t) \times q_s(r)$, which can be rearranged for density $D_s$. Note that through commutativity, we can rearrange these terms as $\mathbb{E}\left[\hat{Y_s}\right] = q_s(r) \times p_s(t) \times A \times D_s$, hence "QPAD" \citep{solymos_calibrating_2013}.

\subsection{Observer Effects} 
Observer effects that seem to go unmodelled but there there 
Observer senescence 
Skill level of IDing birds 
Ability to accurately estimate distance (not good according to (Alldredge et al., 2007) 
Soundscape is getting louder 
Donut effect, esp for grassland birds 

\section{XX Questions for Future Detectability Research}

Shifts in availability over the years? Cite hummingbird example.

Spatial information about detectability? Do we really have to take into account space (either on broad sclae like BCRs or on smaller sclaes like latitude or longitude) when calcualing detectability?

Are avialability and perceptibility truly independent? Some evidence for it, but coudl this be something that is taken into account when using ARU technology?

How good of esitmtaes of detectability will we be able to get from ARUs? There will be different things to consider with ARUs: on availability side of things, the use of a recognizer vs. human listener could affect time to first detection. Can actually get a true "cue rate' with ARU. How accurately can we get true distance to bird form ARUs, without introducing even more sources of error into the mix?

Detectability is hella nuanced, so at what point are we "happy' with a detectability estimate vs when should we continue to potentially torture the data?

Closing remarks 

\section{OLD STUFF }

